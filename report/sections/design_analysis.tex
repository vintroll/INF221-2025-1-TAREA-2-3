\begin{mdframed}
    \textbf{La extensión máxima para esta sección es de 5 páginas.}
\end{mdframed}

Diseñar un algoritmo por cada técnica de diseño de algoritmos mencionada en la sección de objetivos. Cada algoritmo debe resolver el problema de encontrar las diferencias de dos secuencias $s$ y $t$.

\begin{itemize}
    \item Describir la solución diseñada. 
    \item Incluir pseudocódigo (ver ejemplo \cref{alg:mi_algoritmo_1})
    \item Analizar la complejidad temporal y espacial de los algoritmos diseñados en términos de las longitudes de las secuencias de entrada $s$ y $t$.
\end{itemize}

Los pseudocódigos los he diseñado utilizando el paquete \citetitle{algorithm2e} \cite{algorithm2e} para la presentación de algoritmos. Se recomienda consultar \citetitle{ctan-algorithm2e} \cite{ctan-algorithm2e} y \citetitle{overleaf-algorithms} \cite{overleaf-algorithms}.

\begin{mdframed}
    Recuerde que lo importante es diseñar algoritmos que cumplan con los paradigmas especificados. 
\end{mdframed}

\begin{mdframed}
    Si se utiliza algún código, idea, o contenido extraído de otra fuente, este \textbf{debe} ser citado en el lugar exacto donde se utilice, en lugar de mencionarlo al final del informe. 
\end{mdframed}